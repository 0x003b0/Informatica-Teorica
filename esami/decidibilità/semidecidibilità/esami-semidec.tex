\documentclass{article}  % Tipo di documento
\usepackage[utf8]{inputenc} % Per caratteri accentati
\usepackage[italian]{babel} % Lingua italiana
\usepackage{amsthm}
\usepackage{amssymb}
\usepackage{amsmath}  % simboli matematici avanzati
\usepackage{xcolor} % Per i colori
\usepackage{titlesec} % Per personalizzare i titoli
\usepackage{tikz}
\usetikzlibrary{mindmap,trees}
\usepackage[most]{tcolorbox}
\tcbuselibrary{theorems}
\usetikzlibrary{automata, positioning, arrows}
\tcbuselibrary{breakable}
\usepackage{graphicx}

\newtcolorbox{esercizio}[1][]{
    colback=white,       % colore di sfondo
    colframe=green!60!black,       % colore del bordo
    fonttitle=\bfseries,
    title=#1,
    boxrule=0.5pt,       % spessore del bordo
    arc=4pt,             % angoli arrotondati
    left=4pt, right=4pt, top=4pt, bottom=4pt,
    breakable,            % permette di spezzare il box su più pagine
    enhanced,
    break at=0pt
}

% Creo un nuovo ambiente "ragionamento" senza quadratino
\newenvironment{ragionamento}[1][]
  {\begin{proof}[Ragionamento#1]\renewcommand{\qedsymbol}{}\normalfont}
  {\end{proof}}

\title{Esami Informatica Teorica \\ Semidecidibilità}
\author{Ede Boanini}
\date{}  % nessuna data

\begin{document}
\maketitle
\begin{esercizio}[Esercizio 1]
Sia $\Sigma=\{0,1\}$ e sia $R$ un linguaggio decidibile su $\Sigma \times \Sigma$. \\ Dimostra che il 
linguaggio $L=\{x \mid \exists y : (x,y) \in R\}$ è semidecidibile. \\
\begin{ragionamento}
Allora $\Sigma^* \times \Sigma^* = \{(x,y) \mid x\in \Sigma, y \in \Sigma\}$ quindi: \\
\[
R = \{(0, 0), (0, 1), (1, 0), (1, 1)\} \quad \text{lughezza fissa 2; } |x|+|y|=2
\]
\[
L = \{\}
\]
L accetta tutte le stringhe ...
\end{ragionamento}
\end{esercizio}

\begin{esercizio}[Esercizio 2]
Sia $\Sigma=\{0,1\}$ e sia $L$ un linguaggio semidecidibile su $\Sigma$. \\ Dimostrare che esiste un linguaggio
decidibile  $R$  sull'alfabeto $\Sigma \times \Sigma$ tale che $x \in L$ se e solo se $\exists y : (x,y) \in R$.\\
\end{esercizio}

\begin{esercizio}[Esercizio 3]
Si considerino i linguaggi:
\[
    L_{\emptyset}=\{R(M) \mid L(M)=\emptyset\}
\]
\[
    \overline{L_{\emptyset}}=\{R(M) \mid L(M) \neq \emptyset\}
\]
\begin{enumerate}
    \item $\overline{L_{\emptyset}}$ è semidecidibile? Giustificare la risposta.
    \item $L_{\emptyset}$ è semidecidibile? Giustificare la risposta.
\end{enumerate}
\end{esercizio}

\begin{esercizio}[Esercizio 4]
Il complementare di un linguaggio semidecidibile é semidecidibile: vero
o falso? Giustificare la risposta.
\end{esercizio}

\begin{esercizio}[Esercizio 5]
Dati due linguaggi $L_1, L_2$, chiamiamo differenza di $L_1$ e $L_2$ il linguaggio: \\
$L_1 \ominus L_2 = \{vw \mid \exists y \in L_2 : vyw \in L_1\} $. Dimostrare che, se $L_1$ ed $L_2$
sono entrambi semidecidibili, allora $L_1 \ominus L_2$ é anch'esso semidecidibile.
\end{esercizio}

\begin{esercizio}[Esercizio 6]
Per ciascuno dei seguenti problemi, stabilire se esso è semidecidibile oppure no, giustificando la risposta.
Si supponga che l'alfabeto di tutte le macchine di Turing sia \{0,1\}. 
\begin{enumerate}
    \item Date due MdT $M_1, M_2$ che terminano su ogni input, determinare se $M_1$ e $M_2$ 
    accettano linguaggi differenti.
    \item Date due MdT $M_1, M_2$ che terminano sullo stesso insieme di input, determinare
    se $M_1, M_2$ accettano linguaggi differenti.
    \item Date due MdT $M_1, M_2$, determinare se $M_1, M_2$ accettano linguaggi differenti.
\end{enumerate}
\end{esercizio}

\begin{esercizio}[Esercizio 7]
Dimostrare che, per ogni linguaggio semidecidibile $L$, esiste una riduzione
 di $L$ al linguaggio $L_{HALT}$ del problema dell'arresto (Si rammenti che
 $L_{HALT} = \{(R(M),w) \mid M \text{ é una macchina di Turing che accetta } w\}$).
\end{esercizio}

\begin{esercizio}[Esercizio 8]
Sia $Q$ un linguaggio semidecidibile, e sia $L$ un linguaggio t.c. esiste una riduzione 
da $L$ a $Q$. Dimostrare che anche $L$ è semidecidibile.
\end{esercizio}

\begin{esercizio}[Esercizio 9]
Un linguaggio $L$ si dice co-semidecidibile quando il suo complementare $\overline{L}$
è semidecidibile. Sia $L_{\emptyset}=\{R(M) \mid L(M)=\emptyset\}$. \\
Dimostrare che $L_{\emptyset}$ è co-semidecidibile.
\end{esercizio}
\end{document}