\documentclass{article}  % Tipo di documento
\usepackage[utf8]{inputenc} % Per caratteri accentati
\usepackage[italian]{babel} % Lingua italiana
\usepackage{amsthm}
\usepackage{amssymb}
\usepackage{amsmath}  % simboli matematici avanzati
\usepackage{xcolor} % Per i colori
\usepackage{titlesec} % Per personalizzare i titoli
\usepackage{tikz}
\usetikzlibrary{mindmap,trees}
\usepackage[most]{tcolorbox}
\tcbuselibrary{theorems}
\usetikzlibrary{automata, positioning, arrows}
\tcbuselibrary{breakable}
\usepackage{graphicx}

\newtcolorbox{esercizio}[1][]{
    colback=white,       % colore di sfondo
    colframe=green!60!black,       % colore del bordo
    fonttitle=\bfseries,
    title=#1,
    boxrule=0.5pt,       % spessore del bordo
    arc=4pt,             % angoli arrotondati
    left=4pt, right=4pt, top=4pt, bottom=4pt,
    breakable,            % permette di spezzare il box su più pagine
    enhanced,
    break at=0pt
}

\title{Esame Informatica Teorica \\ 12 gennaio 2016}
\author{Ede Boanini}
\date{}  % nessuna data

\begin{document}
\maketitle
\begin{esercizio}[Esercizio 1]
\begin{enumerate}
    \item Sia $\Sigma=\{0,1\}$ e sia $R$ un linguaggio decidibile su $\Sigma \times \Sigma$. \\ Dimostra che il 
    linguaggio $L=\{x \mid \exists y : (x,y) \in R\}$ è semidecidibile.
\end{enumerate}
Allora $\Sigma^* \times \Sigma^* = \{(x,y) \mid x\in \Sigma, y \in \Sigma\}$ quindi: \\
\[
R = \{(0, 0), (0, 1), (1, 0), (1, 1)\} \quad \text{lughezza fissa 2; } |x|+|y|=2
\]
\[
L = \{\}
\]
L accetta tutte le stringhe ...
\end{esercizio}
\end{document}