\documentclass{article}  % Tipo di documento
\usepackage[utf8]{inputenc} % Per caratteri accentati
\usepackage[italian]{babel} % Lingua italiana
\usepackage{amsthm}
\usepackage{amssymb}
\usepackage{amsmath}  % simboli matematici avanzati
\usepackage{xcolor} % Per i colori
\usepackage{titlesec} % Per personalizzare i titoli
\usepackage{tikz}
\usetikzlibrary{mindmap,trees}
\usepackage[most]{tcolorbox}
\tcbuselibrary{theorems}
\usetikzlibrary{automata, positioning, arrows}
\tcbuselibrary{breakable}
\usepackage{graphicx}

\newtcolorbox{esercizio}[1][]{
    colback=white,       % colore di sfondo
    colframe=green!60!black,       % colore del bordo
    fonttitle=\bfseries,
    title=#1,
    boxrule=0.5pt,       % spessore del bordo
    arc=4pt,             % angoli arrotondati
    left=4pt, right=4pt, top=4pt, bottom=4pt,
    breakable,            % permette di spezzare il box su più pagine
    enhanced,
    break at=0pt
}

\title{Esame Informatica Teorica \\ 16 giugno 2016}
\author{Ede Boanini}
\date{}  % nessuna data

\begin{document}
\maketitle
\begin{esercizio}[Esercizio 1]
Dati due linguaggi decidibili $L_1$ e $L_2$, dimostrare che:
\begin{enumerate}
    \item L'unione $L_1 \cup L_2$ è un linguaggio decidibile
    \item La concantenazione $L_1L_2$ è un linguaggio decidibile
    \item La chiusura di Kleene $L_1^*$ è un linguaggio ricorsivo
\end{enumerate}
Ricordando che:
\begin{align*}
    L_1 \cup L_2 = \{w \mid w \in L_1 \text{ oppure } w \in L_2\}
\end{align*}
\begin{align*}
    L_1L_2= \{w \mid \exists w_1 \in L_1, w_2 \in L_2 : w=w_1w_2\}
\end{align*}
\begin{align*}
    L_1^*= \{\text{solo concantenazione di stringhe che già appartengona a } L_1\}
\end{align*}





\end{esercizio}
\end{document}