\documentclass{article}  % Tipo di documento
\usepackage[utf8]{inputenc} % Per caratteri accentati
\usepackage[italian]{babel} % Lingua italiana
\usepackage{amsthm}
\usepackage{amssymb}
\usepackage{amsmath}  % simboli matematici avanzati
\usepackage{xcolor} % Per i colori
\usepackage{titlesec} % Per personalizzare i titoli
\usepackage{tikz}
\usetikzlibrary{mindmap,trees}
\usepackage[most]{tcolorbox}
\tcbuselibrary{theorems}
\usetikzlibrary{automata, positioning, arrows}
\tcbuselibrary{breakable}
\usepackage{graphicx}

\newtcolorbox{esercizio}[1][]{
    colback=white,       % colore di sfondo
    colframe=green!60!black,       % colore del bordo
    fonttitle=\bfseries,
    title=#1,
    boxrule=0.5pt,       % spessore del bordo
    arc=4pt,             % angoli arrotondati
    left=4pt, right=4pt, top=4pt, bottom=4pt,
    breakable,            % permette di spezzare il box su più pagine
    enhanced,
    break at=0pt
}

\title{Esami Informatica Teorica \\ Indecidibilità}
\author{Ede Boanini}
\date{}  % nessuna data

\begin{document}
\maketitle
\begin{esercizio}[Esercizio 1]
 Sia  $M$  una MdT deterministica che accetta un linguaggio non decidibile (non ricorsivo) per stati finali. 
 Dimostrare che il problema dell'arresto per  $M$
 non è decidibile, vale a dire, che non esiste una macchina di Turing che, presa in
 input una stringa  $w$ , determina (decide) se la computazione di  $M$  con input  $w$
 termina oppure no.
\end{esercizio}

\begin{esercizio}[Esercizio 2]
 Sia  $L$  il linguaggio costituito dalle terne  $(R(M),w, q)$ tali che  $R(M)$  è la
 rappresentazione di una macchina di Turing  $M$  che, fatta partire su input  $w$, termina
 nello stato  $q$. Dimostrare che  $L$  non è decidibile.
\end{esercizio}

\begin{esercizio}[Esercizio 3]
 Si consideri il seguente problema: \\
 Data una macchina di Turing  $M$, determinare se esiste una stringa  $w$
 sulla quale  M  termina. \\
 Dimostrare che tale problema è indecidibile.
\end{esercizio}

\begin{esercizio}[Esercizio 4]
Si consideri il seguente problema: \\
 Data una macchina di Turing a due nastri $M$ e una stringa $w$, determinare
 se $M$, fatta partire su $w$ scritta sul primo nastro, nel corso dell'esecuzione
 compie almeno unoperazione di scrittura sul secondo nastro. \\
 Dimostrare che tale problema e indecidibile.
\end{esercizio}

\begin{esercizio}[Esercizio 5]
Si consideri il seguente problema: \\
Date due macchine di Turing $M_1, M_2$, determinare se il linguaggio accettato da $M_1$ è uguale al linguaggio accettato da $M_2$. \\
\begin{enumerate}
    \item Descrivere il linguaggio formale associato a tale problema
    \item Dimostrare che tale problema è indecidibile
\end{enumerate}
\end{esercizio}

\begin{esercizio}[Esercizio 6]
    Per ciascuno dei seguenti problemi, stabilire se esso è decidibile o indecidibile, giustificando la risposta.
    Si supponga che l'alfabeto di tutte le macchine di Turing sia \{0,1\}.
    \begin{enumerate}
        \item Date due MdT $M_1, M_2$, determinare se $M_1$ e $M_2$ accettano lo stesso linguaggio.
        \item Date due MdT $M_1, M_2$, che terminano su ogni input, determinare se $M_1$ e $M_2$
        accettano lo stesso linguaggio
        \item Date due MdT $M_1, M_2$, che terminano su ogni input, determinare se $M_1$ e $M_2$
        accettano le stesse stringhe di lunghezza al piú 100 ($|w| \leq 100$).
    \end{enumerate}
\end{esercizio}

\begin{esercizio}[Esercizio 7]
Si consideri il seguente problema: \\
 Data una macchina di Turing $M$, determinare se il linguaggio $L(M)$
 accettato da $M$ ha la seguente proprieta: ogni volta che la stringa
 $w =w_1w_2 \cdots w_n \in L(M)$, anche la stringa rovesciata
 $w^R =w_nw_{n-1} \cdots w_1 \in L(M)$. \\
 Dimostrare che tale problema e indecidibile.
\end{esercizio}

\begin{esercizio}[Esercizio 8]
    Dimostrare ch il linguaggio $L_{Halt}$ del problema dell'arresto non è riducibile
    a $L_{\emptyset}$. \\
    Ricordo che:
    \[
    L_{Halt}=\{R(M) \mid M \text{ termina su } w\}
    \]
    e che 
    \[
    L_{\emptyset} = \{R(M) \mid L(M)=\emptyset\}
    \]
\end{esercizio}

\begin{esercizio}[Esercizio 9]
    Una MdT $M$ si dice riproducibile quando esiste un'altra MdT $M'$ che accetta lo stesso 
    linguaggio di $M$. Indichiamo con $L(M)$ il linguaggio (semidecidibile) accettato da una 
    generica MdT $M$. Stabilire se ciascuno dei seguenti problemi è decidibile o indecidibile, giustificando la risposta.
    \begin{enumerate}
        \item Dato un linguaggio semidecidibile $L$, determinare se esiste una MdT riproducibile $M$ t.c. $L=L(M)$.
        \item Dato un linguaggio semidecidibile $L$, determinare se esiste una MdT riproducibile $M$ avente meno di 10 stati t.c. $L=L(M)$.
        \item Dato un linguaggio semidecidibile $L$, determinare se esiste una MdT riproducibile $M$ avente più di 10 stati t.c. $L=L(M)$
    \end{enumerate}
\end{esercizio}

\end{document}